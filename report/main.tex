\documentclass[a4paper]{article}

\usepackage[utf8]{inputenc}
\usepackage[T2A]{fontenc}
\usepackage[russian]{babel}

\usepackage{mathtext}

\usepackage{listings}
\usepackage{color}

\definecolor{mygreen}{rgb}{0,0.6,0}
\definecolor{mygray}{rgb}{0.5,0.5,0.5}
\definecolor{mymauve}{rgb}{0.58,0,0.82}

\lstset{
	morekeywords={*,...}, % если хотите добавить ключевые слова
	keywordstyle=\color{blue},
	stringstyle=\color{mymauve},
	% Настройки отображения
	breaklines=true, % Перенос длинных строк
	basicstyle=\footnotesize, % Шрифт для отображения кода
	backgroundcolor=\color{white}, % Цвет фона кода
	frame=single,
	rulecolor=\color{black}, % Цвет рамки
	tabsize=3, % Размер табуляции в пробелах
	% Настройка отображения номеров строк. Если не нужно, то удалите весь блок
	numbers=left, % Слева отображаются номера строк
	stepnumber=1, % Каждую строку нумеровать
	numbersep=5pt, % Отступ от кода
	numberstyle=\tiny\color{mygray},
	% Для отображения русского языка
	extendedchars=true,
	literate={Ö}{{\"O}}1
	{Ä}{{\"A}}1
	{Ü}{{\"U}}1
	{ß}{{\ss}}1
	{ü}{{\"u}}1
	{ä}{{\"a}}1
	{ö}{{\"o}}1
	{~}{{\textasciitilde}}1
	{а}{{\selectfont\char224}}1
	{б}{{\selectfont\char225}}1
	{в}{{\selectfont\char226}}1
	{г}{{\selectfont\char227}}1
	{д}{{\selectfont\char228}}1
	{е}{{\selectfont\char229}}1
	{ё}{{\"e}}1
	{ж}{{\selectfont\char230}}1
	{з}{{\selectfont\char231}}1
	{и}{{\selectfont\char232}}1
	{й}{{\selectfont\char233}}1
	{к}{{\selectfont\char234}}1
	{л}{{\selectfont\char235}}1
	{м}{{\selectfont\char236}}1
	{н}{{\selectfont\char237}}1
	{о}{{\selectfont\char238}}1
	{п}{{\selectfont\char239}}1
	{р}{{\selectfont\char240}}1
	{с}{{\selectfont\char241}}1
	{т}{{\selectfont\char242}}1
	{у}{{\selectfont\char243}}1
	{ф}{{\selectfont\char244}}1
	{х}{{\selectfont\char245}}1
	{ц}{{\selectfont\char246}}1
	{ч}{{\selectfont\char247}}1
	{ш}{{\selectfont\char248}}1
	{щ}{{\selectfont\char249}}1
	{ъ}{{\selectfont\char250}}1
	{ы}{{\selectfont\char251}}1
	{ь}{{\selectfont\char252}}1
	{э}{{\selectfont\char253}}1
	{ю}{{\selectfont\char254}}1
	{я}{{\selectfont\char255}}1
	{А}{{\selectfont\char192}}1
	{Б}{{\selectfont\char193}}1
	{В}{{\selectfont\char194}}1
	{Г}{{\selectfont\char195}}1
	{Д}{{\selectfont\char196}}1
	{Е}{{\selectfont\char197}}1
	{Ё}{{\"E}}1
	{Ж}{{\selectfont\char198}}1
	{З}{{\selectfont\char199}}1
	{И}{{\selectfont\char200}}1
	{Й}{{\selectfont\char201}}1
	{К}{{\selectfont\char202}}1
	{Л}{{\selectfont\char203}}1
	{М}{{\selectfont\char204}}1
	{Н}{{\selectfont\char205}}1
	{О}{{\selectfont\char206}}1
	{П}{{\selectfont\char207}}1
	{Р}{{\selectfont\char208}}1
	{С}{{\selectfont\char209}}1
	{Т}{{\selectfont\char210}}1
	{У}{{\selectfont\char211}}1
	{Ф}{{\selectfont\char212}}1
	{Х}{{\selectfont\char213}}1
	{Ц}{{\selectfont\char214}}1
	{Ч}{{\selectfont\char215}}1
	{Ш}{{\selectfont\char216}}1
	{Щ}{{\selectfont\char217}}1
	{Ъ}{{\selectfont\char218}}1
	{Ы}{{\selectfont\char219}}1
	{Ь}{{\selectfont\char220}}1
	{Э}{{\selectfont\char221}}1
	{Ю}{{\selectfont\char222}}1
	{Я}{{\selectfont\char223}}1
	{і}{{\selectfont\char105}}1
	{ї}{{\selectfont\char168}}1
	{є}{{\selectfont\char185}}1
	{ґ}{{\selectfont\char160}}1
	{І}{{\selectfont\char73}}1
	{Ї}{{\selectfont\char136}}1
	{Є}{{\selectfont\char153}}1
	{Ґ}{{\selectfont\char128}}1
	{\{}{{{\color{black}\{}}}1 % Цвет скобок {
	{\}}{{{\color{black}\}}}}1 % Цвет скобок }
	,
%	title=\lstname
}

\usepackage{algorithmicx}
\usepackage{algpseudocode}

\usepackage{amssymb}
\usepackage{amsopn}
\usepackage{mathtools}

\usepackage{graphicx}

\usepackage[
a4paper, includefoot,
left=2cm, right=1cm, top=2cm, bottom=2cm,
]{geometry}

\usepackage[hidelinks]{hyperref}
\usepackage{multirow}
\usepackage{cmap}


\begin{document}

\begin{titlepage}

	\begin{center}

		\large Федеральное государственное автономное образовательное учреждение высшего образования \\
		\large «Санкт-Петербургский политехнический университет Петра Великого» \\
		\large Институт компьютерных наук и технологий \\
		\large Кафедра «Компьютерные интеллектуальные технологии» \\[4cm]

		\huge {\bf Курсовая работа} \\[0.5cm]
		\large {\bf Информационная система автомобилестроительного предприятия} \\[0.1cm]
		\large по дисциплине «Базы данных» \\[4cm]

	\end{center}

    \begin{center}
        \begin{minipage}[t]{4cm}
            \begin{flushleft}
                Выполнил студент гр. 23506/1
            \end{flushleft}
        \end{minipage}
        \hfill
        \begin{minipage}[t]{4cm}
            \begin{flushright}
            О.Д. Романов
            \end{flushright}
        \end{minipage} \\[0.5cm]

        \begin{minipage}[t]{4cm}
            \begin{flushleft}
                Руководитель старший преподаватель
            \end{flushleft}
            \flushleft
        \end{minipage}
        \hfill
        \begin{minipage}[t]{4cm}
            \begin{flushright}
                Н.В. Андреева
            \end{flushright}
        \end{minipage}
    \end{center}

    \begin{flushright}
        14 мая 2017
    \end{flushright}

	
	\vfill

	\begin{center}
	    \large Санкт-Петербург\\
	    \large \the\year
	\end{center}
 
\end{titlepage}

\tableofcontents
\newpage

\section{Исходные данные}

\subsection{Техническое задание}
Структурно предприятие состоит из цехов, которые в свою очередь подразделяются на участки.

Категории изделий, выпускаемых предприятием: грузовые, легковые автомобили, автобусы, сельскохозяйственные, дорожно-строительные машины, мотоциклы и прочие изделия.
Каждая категория изделий имеет специфические, присущие только ей атрибуты.
Например, для автобусов это вместимость, для сельскохозяйственных и дорожно-строительных машин - производительность и т.д.

По каждой категории изделий может собираться несколько видов изделий (под видом изделия понимается конкретная его разновидность / марка - например, автомобиль KIA Rio).
По конкретным экземплярам каждого вида ведётся журнал, где отмечаются даты завершения различных этапов жизненного цикла изделия: изготовление (сборка) / тестирование / передача дилеру / гарантийный ремонт.

Предприятие в основном состоит из производственных цехов, но также есть несколько вспомогательных (например, ремонтный, тестировочный).

Каждая категория изделий собирается в своём производственном цехе (в одном цехе может собираться несколько категорий изделий).
Цех структурно состоит из участков, на каждом из которых выполняется один вид работ: изготавливается определённая часть изделия (например, двигатель) либо производится сборка изделия в целом.
С каждой категорией изделия ассоциируется свой набор работ; другими словами, каждая категория в процессе изготовления должна пройти определённый набор участков в цехе.

Каждой категории инженерно-технического персонала (инженеры, технологи, техники) и рабочих (сборщики, токари, слесари, сварщики и пр.) также характерны атрибуты, свойственные только для этой группы.
Рабочие объединяются в бригады, которыми руководят бригадиры.
Бригадиры выбираются из числа рабочих.
Работу цеха возглавляет начальник цеха, а работу на участке - начальник участка, в подчинении которого находится несколько мастеров.
Каждый мастер координирует работу одной или нескольких бригад (но, в отличие от бригадира, не входит в состав конкретной бригады).
Мастера, начальники участков и цехов назначаются из числа инженерно-технического персонала.
Каждый начальник может руководить только одной структурной единицей (в т.ч. начальник одной структурной единицы не может быть в то же время начальником другой).

Работу по сборке конкретной категории изделия на определенном участке выполняет одна бригада рабочих, при этом она может обслуживать несколько участков / категорий и на одном участке может работать несколько бригад.

\subsection{Виды запросов в информационной системе}
\begin{enumerate}

    \item Перечень видов изделий по категории, собираемой указанным цехом.
    В последней строке вывести общее число собираемых видов изделий.

    \item Количество экземпляров изделий каждого вида каждой категории, собранных предприятием за определенный отрезок времени.
    В последней строке вывести общее число собранных изделий.
    Примерный вид результата:

    \begin{tabular}{|p{6cm}|p{2cm}|p{1cm}|} \hline
        Категория & Вид & Кол-во \\ \hline
        Автобусы & АКБ-12 & 8 \\ \hline
        Автобусы & АКБ-05 & 0 \\ \hline
        Автомобили & ИЖ-400 & 12 \\ \hline
        ... & ... & ... \\ \hline
        Всего собрано (12.05.2010-18.07.2010): & & 48 \\ \hline
    \end{tabular}

    \item Данные о кадровом составе (ФИО, должность) по указанным категориям инженерно-технического персонала и рабочих;
    \item Число и перечень участков предприятия и их начальников (с указанием цехов).
    \item Перечень работ, которые проходит указанный вид изделия.
    \item Состав бригад, работающих на указанном участке указанного цеха: ФИО рабочего, номер бригады, номер участка, номер цеха. Отсортировать по номеру бригады.
    \item Перечень мастеров (ФИО) указанного участка указанного цеха и номера бригад, работы которых они координируют.
    \item Информация о цехах, в которых в настоящий момент собирается больше видов изделий, чем в среднем приходится на каждый производственный цех предприятия: номер цеха, название цеха, кол-во собираемых видов изделий, среднее количество видов изделий по цехам предприятия.
    \item Состав бригад, участвующих в сборке указанной категории изделия.
    \item ФИО и должности работников цеха, в котором собирается больше всего категорий изделий.

\end{enumerate}

\section{Сущности, их харрактеристики и связи}
В ходе анализа начального описания предметной области были выявлены следующие сущности:

\begin{enumerate}

    \item

\end{enumerate}

\end{document}